
% ----------------------------------------------------------
% DADOS DO TRABALHO - CAPA e FOLHA DE ROSTO
% Configure os dados do trabalho aqui
% ----------------------------------------------------------
\titulo{\textbf{Isolating Feature Effects in Supervised Machine Learning Illustrated in Educational Data Mining}}
\autor{Rogério Luiz Cardoso Silva Filho}
\local{Recife}
\data{\Year}
\areaconcentracao{\textbf{Área de Concentração}: Inteligência Artificial}
\orientador{\textbf{Orientador (a)}: Paulo Jorge Leitão Adeodato}
\coorientador{\textbf{Coorientador (a)}: Kellyton dos Santos Brito}

\instituicao{UNIVERSIDADE FEDERAL DE PERNAMBUCO \\ CENTRO DE INFORMÁTICA \\PROGRAMA DE PÓS-GRADUAÇÃO EM CIÊNCIAS DA COMPUTAÇÃO}
\departamento{Centro de Informática}
\programa{Pós-graduação em Ciências da Computação}
\emailprograma{rlcsf@cin.ufpe.br}
\siteprograma{http://cin.ufpe.br/\textasciitilde posgraduacao}

\tipotrabalho{phdthesis}
% O preambulo deve conter o tipo do trabalho, o objetivo, 
% o nome da instituição e a área de concentração 
%\preambulo{Trabalho apresentado ao Programa de Pós-graduação em Ciência da Computação do Centro de Informática da Universidade Federal de Pernambuco, como requisito parcial para obtenção do grau de Doutor em Ciência da Computação.}

%\preambuloatadefesa{Dissertação apresentada ao Programa de Pós-Graduação Profissional em Ciência da Computação da Universidade Federal de Pernambuco, como requisito parcial para a obtenção do título de Doutor em 04 de setembro de 2020.}

\preambulo{Trabalho apresentado ao Programa de Pós-graduação em Ciência da Computação do Centro de Informática da Universidade Federal de Pernambuco, como requisito parcial para obtenção do grau de Doutor em Ciências da Computação}

\preambuloatadefesa{Texto texto texto texto texto texto texto texto texto texto texto texto texto texto texto texto texto texto texto texto texto texto texto texto texto texto texto texto texto texto texto texto.}




\input{userlists}



