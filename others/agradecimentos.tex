% ----------------------------------------------------------
% AGRADECIMENTOS
% ----------------------------------------------------------
\begin{agradecimentos}
Concluir um doutorado é como alcançar o cume de uma imensa montanha. A sensação de dever cumprido ao olhar para baixo e perceber toda a trilha percorrida, as paradas de descanso e tudo aquilo que se une para levar ao topo reflete bem a jornada de um doutorado. Ambos começam com um sonho, sereno, que vai se complicando e oferecendo inúmeras oportunidades de desistir, mas, ao final, fazem você refletir orgulhoso: "Como pude chegar até aqui?" 

É neste momento que me encontro, e mesmo tendo passado grande parte desses últimos anos sozinho em frente a um computador, tenho convicção de que só pude chegar aqui por estar sendo apoiado por muitos. Como em uma montanha, onde, apesar do esforço individual, o alcance do topo normalmente depende de uma equipe de guias, cozinheiros e carregadores. 

Durante minha jornada acadêmica, tive o privilégio de contar com uma equipe de apoio formada por pessoas e instituições. Sou imensamente grato pelo apoio familiar de Maria Alice e Antônio, pela base construída por meus pais Rogério e Madalena, por minha avó Joana e tia Enedina (\textit{in memoriam}) e pela irmandade de Ana Caroline e Daniel.

Agradeço ao professor Paulo Adeodato, um grande pensador que me guiou. Seus ensinamentos me acompanharão por toda a vida. Meu reconhecimento também ao professor Kellyton, um colega que se tornou coorientador. Aos amigos e a todos que torceram por mim, essa conquista também tem uma parte de vocês.

Também sou grato ao IFNMG. Agradeço o apoio da gestão e dos colegas de trabalho, que corajosamente puderam absorver minhas responsabilidades. Sou grato à Fundação Lemann por financiar meu intercâmbio e me permitir conhecer professores que muito contribuíram para minha formação, como Martin Carnoy e Eric Bettinger. 

Minha gratidão também vai para todos os educadores - professores, bibliotecários, escritores, supervisores e funcionários - dos ambientes educacionais que frequentei ao longo da vida, que, direta ou indiretamente, me inspiraram a acreditar no caminho da educação formal. 

Agora, é hora de descer, o que também exigirá certo esforço, enquanto os pensamentos sobre os próximos desafios começam a surgir. Da montanha, leva-se algumas fotos que se tornarão pequenas diante de todo o aprendizado acumulado na subida. Da mesma forma, espero que a minha nova formação vá muito além do título e que possa ser útil na construção de uma sociedade melhor e mais justa.

\end{agradecimentos}