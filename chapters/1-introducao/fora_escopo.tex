\section{Out of Scope}
\label{fora_escopo}

Since this thesis encompasses a broad context, it is important to highlight a set of subjects that are outside the scope of this thesis:

\begin{itemize}
    \item \textbf{Introduce new strategy to deal with the extrapolation issue}. The extrapolation problem is a well-known issue in the literature and is sometimes even considered a trade-off, where it is not possible to be fully adherent to the data and model simultaneously \cite{Lundberg2020FromTrees., Chen2020TrueData}. Rather than proposing novel strategies, this thesis focuses on discussing and expanding the understanding and application of existing strategies within this context. The primary focus is on extracting knowledge from data.

    \item \textbf{Guidelines for Explaining EDM Models}: This thesis advocates for the ALE framework as a favorable and appropriate alternative for globally elucidating the role of features in predictive models, especially within educational contexts dealing with correlated data. However, it does not intend to set rigid guidelines for explaining EDM models. As discussed in Chapter 2, explanations can vary both in form and function, tailored to suit the explainability requirements of the intended audience. The diversity of these explanations and their integration can substantially enhance knowledge discovery.

    \item \textbf{Compromise with causality}.  While the primary aim is to provide dependable insights into the mechanisms generating the underlying data, it's important to acknowledge that traditional supervised models do not ensure an accurate reconstruction of the data-generating process. Often, these models rely on spurious correlations for making predictions rather than establishing causal relationships. Consequently, this work does not delve into any causality-related issues.
    
\end{itemize}