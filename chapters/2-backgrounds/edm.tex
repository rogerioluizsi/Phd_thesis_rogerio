\section{EDM}

In recent times, a rapidly expanding body of \gls{ML} literature has emerged, introducing a diverse range of new tools, including algorithms, data preprocessing techniques, frameworks, and model validation methods. These tools have been developed to provide support for empirical researchers who utilize data to address a variety of problems \cite{Athey2019MachineAbout}. When these tools are specifically tailored for use with educational data, they serve as the foundation of a growing research area known as \gls{EDM}. \gls{EDM}, as described by \cite{Romero2020EducationalSurvey}, represents an interdisciplinary field dedicated to the analysis of extensive and complex educational datasets, with the goal of building predictions and extracting actionable insights and knowledge to support decision-makers in the realm of education.

The application of \gls{EDM} extends across multiple domains within the educational sector. Much of the research in this domain has been centered on data derived from learning management systems within specific educational institutions \cite{Fischer2020MiningChallenges}, mainly in universities \cite{Romero2020EducationalSurvey}. Research studies in \gls{EDM} encompass various subfields, including the investigation of cognitive strategies \cite{Fancsali2018IntelligentOffs, Moussavi2016TheTopics}, prediction of student dropout \cite{Chaturapruek2018HowGPA, Jayaprakash2014EarlyInitiative}, and the development of intelligent tutoring systems \cite{Jiang2019Goal-basedRecommendation}. A common thread in all these areas is the task of predicting student performance, a task that, despite significant enhancements facilitated by advanced ML algorithms, still requires further advancements in providing explanations for the underlying factors driving these predictions \cite{Yang2021InterpretabilityLearning, Kovalev2020EducationalSolutions}.

The provision of such explanatory insights is critical, as merely presenting probabilities may prove inadequate for enhancing educational systems. For example, in automating an admission system with an \gls{EDM} solution, fully understanding the factors behind these probabilities can greatly improve the admission process. It provides the committee with important information to increase fairness and transparency \cite{AlGhamdi2020APrediction, Maulana2023OptimizingPerspective}. Also, other processes like loan grants can benefit from these explanations \cite{Maulana2023OptimizingPerspective}.

Additionally, there are \gls{EDM} applications specifically designed to extract insights from data, making explanations a key objective. For example, \gls{EDM} has been highly effective in processing large datasets generated by modern \gls{LSA} tests in the field of Educational Assessment \cite{Liu2008UsingEnergy, CardosoSilvaFilho2023BeyondEffectiveness, SilvaFilho2023LeveragingEducation, Saarela2016PredictingApproach}. In these cases, the main goal is to discover knowledge about educational systems, providing crucial insights that support discussions on educational policies and guide the generation of novel hypotheses for subsequent confirmatory work.

In the domain of supervised learning, regression and classification tasks are commonly employed techniques \cite{Aldowah2019EducationalSynthesis}. These tasks utilize algorithms that are universally recognized across various data mining fields, including Support Vector Machines, decision trees and their variations, logistic regression, and neural networks. While there is no consensus on the single most utilized algorithm, tree-based algorithms—especially Random Forest—emerge as a clear preference \cite{Rastrollo-Guerrero2020AnalyzingReview, Khan2021StudentStudies, Martinez-Abad2020EducationalAssessment, Namoun2020PredictingReview}. This preference may be attributed to their widespread implementation across \gls{ML} tools and the straightforward manner in which they allow for the interpretation of feature importance via \gls{MDI} scores





