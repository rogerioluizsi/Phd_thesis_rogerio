\section{Supervised Learning }

Supervised learning is a subfield of \gls{ML} in which a model is trained on a labeled dataset to perform predictive tasks in a unseen dataset. The objective is to find a function \( f: X \rightarrow Y \), where \( X \) is the feature space and \(Y\) is the output space, such that the function approximates the underlying mapping from input features to outputs as closely as possible.

Specifically, given a labeled dataset \(D = {(x_1, y_1), (x_2, y_2), \ldots, (x_n, y_n)} \), where \( x_i \in X \) and \( y_i \in Y \), the supervised learning algorithm aims to minimize a loss function \( L(f(X), Y) \) over \( D\), defined as:

\[
L(f) = \sum_{i=1}^{n} L(f(x_i), y_i)
\]

Where, \(L\) measures the discrepancy between the predicted output \( f(x_i) \) and the true label \( y_i \). To enhance the model's generalization capabilities, $f$ is evaluated on an statistically independent dataset, thereby mitigating the risk of optimistic empirical performance estimation.

\subsection{Contrast with traditional statistical methods}

Despite the models used in traditional statistics can be seen as one of the tools available in \gls{ML} \cite{TrevorHastieRobertTibshirani2014AssessmentSelection}, there is a fundamental difference in how they estimate the weight of functions. To elucidate the distinctions between supervised learning and traditional statistical models, lets consider a linear model within the supervised learning framework, defined as:

\begin{equation}
y = \beta_0 + \beta_1 x + \epsilon
\label{lm}
\end{equation}

In contrast to the Ordinary Least Squares (OLS) approach prevalent in traditional statistical analyses, the objective in the context of supervised learning is also to optimize the coefficients $\beta$ in a manner that minimizes out-of-sample error. Specifically, OLS optimizes \( \beta \) by minimizing only in-sample error, without explicit consideration for out-of-sample generalizability. The key divergence stems from supervised learning's strategic focus on balancing the bias-variance trade-off, thereby allowing a certain level of bias (in-sample error) to mitigate excessive variance (out-of-sample error) \cite{Athey2019MachineAbout}.

The objective function for linear regression under  supervised learning paradigm can be formalized as:

\begin{equation}
\sum_{i=1}^{n} (y_i - f(x_i))^2 + \lambda R(f)
\end{equation}

In this equation, \( (y_i - f(x_i))^2 \), represents the in-sample error, while the regularization term \( R(f) \) acts to prevent overfitting by constraining the model's complexity, thereby reducing the out-of-sample error. The chosen regularization parameter \( \lambda \) plays a crucial role in modulating the extent of this constraint, thereby influencing the model's generalization performance \cite{TrevorHastieRobertTibshirani2014AssessmentSelection}.






