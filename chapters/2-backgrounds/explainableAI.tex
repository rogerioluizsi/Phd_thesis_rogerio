\section{XAI}

The wide use of \gls{AI} and \gls{ML} has increasingly emphasized the importance of transparency and user comprehension of model behaviors, primarily under the terms Explainable AI (\gls{XAI}) and \gls{IML}. Despite the significant growth of this research area in recent years \cite{Arya2019OneTechniques}, foundational works in the field can be traced back to the 1980s \cite{Fagant1980COMPUTER-BASEDVM, Bareiss1988Protos:Apprentice}. While some authors argue that \gls{XAI} and \gls{IML} can be conceptually distinct \cite{Watson2022ConceptualLearning}, they are more commonly used interchangeably in the broader scientific literature as both terms share the objective of enhancing the transparency of \gls{ML} models \cite{Molnar2022Model-agnosticLearning}

In this thesis, the term "interpretability" will used to refer to its common dictionary meaning, while "explainability" will be specifically employed to describe the systematic extraction of knowledge about predictive models. 

\subsection{Inherent interpretable models}

Inherent interpretable models, often referred to as intrinsically interpretable models, are those models distinguished by their transparent and easily understandable internal mechanics. These model provide explicit explanations of the relationships between input features and output predictions, facilitating a deeper understanding of their recommendations in the decision-making processes. 

Examples of such models include linear regression, decision trees, and induction rules. In the linear regression \ref{lm}, for instance, each $\beta$ coefficient quantifies the change in a dependent variable for a one-unit change in an independent variable, assuming all other variables are held constant. In other words, the model additive parametrization allows an isolated interpretation of the effects of individual features. Many other adaptions allow a linear model to capture more complex relationships \cite{TrevorHastieRobertTibshirani2014AssessmentSelection}, such as interactions and non-linearity. Nevertheless, in models that involve transformations of this linear predictor into other discrete outcomes, such as in the logit and probit models, the $\beta$ interpretation is not straightforward and limited \cite{Mood2017LogisticHeterogeneity, long1997regression}. 

The decision trees categorize outcomes based on decision rules at each node. On the other hand, inducing rules out of tree structure do not narrow the dimensional space as it occurs in the trees. These induced rules are clear statements in the natural language of how inputs lead to outputs in several different perspectives \cite{SilvaFilho2019DataInstitutes}.

The primary advantage of using inherent interpretable models is their ease of interpretation, which is especially beneficial to high-stakes decisions \cite{Rudin2019StopInstead}. However, they are often outperformed by more complex models when it comes to predictive performance \cite{Loyola-Gonzalez2019Black-boxView}. The simplicity that makes them easy to interpret can also be a drawback, as it might lead to the oversimplification of intricate relationships in the data. This can be a significant limitation when dealing with complex systems where multiple variables interact nonlinearly.

\subsection{Post-hoc explainable techniques}

When a second model is used to explain the first, it is categorized as a post-hoc explainable technique. Model-agnostic explanation model is any function \(g\) that approximates the original model \(f\) \cite{10.5555/3295222.3295230}. While intrinsically interpretable models provide insights into predictions using their internal components, post-hoc techniques treat models as opaque, relying solely on their prediction function and data \cite{Molnar2022Model-agnosticLearning}. 

Post-hoc techniques are typically model-agnostic and offer the flexibility to explain various types of models, including those that are transparent, in an effort to enhance existing explanations. For instance, in a logistic regression model, a model-agnostic technique can illustrate the individual feature effects across the entire range of the feature values, whereas the coefficients only indicate the order of feature contributions. Similarly, understanding feature effects can complement the intrinsic rule-based explanations provided by a decision tree. 

\subsection{Measuring explainability}

Explainability is a domain-specific notion and has a big criticism of the lack of formalization \cite{Rudin2019StopInstead,Watson2022ConceptualLearning,Lipton2018TheInterpretability}. Explanations can take various forms, and there isn't a clear definition of what constitutes an explanation. Moreover, explanations can differ based on the type of input variables. For images, explanations are often visualized as heatmaps, whereas for text inputs, they typically involve highlighting text passages or emphasizing words \cite{Molnar2022Model-agnosticLearning}.

This thesis focuses on tabular data. For certain tasks within this domain, visual graphs by using plots may be the preferred form of explanation, while others might favor text or scores. This variation makes it challenging to find a widely accepted definition of explainability, even within this narrowed scope. Such diversity in explanations presents a challenge in defining quantifiable evidence for the field \cite{Doshi-Velez2017TowardsLearning}. 

Unlike supervised learning, where much of the literature has advanced based on clear benchmarks of model performance, the sub-field of \gls{XAI} or \gls{IML} still faces vagueness in definitions. This is due to the challenge of measuring the trustworthiness of model explanations, as there is no ground truth for comparison in the real world which is only known by its observable data. Determining which explanation is superior is also difficult \cite{Arya2019OneTechniques}, even for inherently interpretable models. For instance, we can't always say whether a decision tree path is more or less clear than a linear model's coefficients \cite{Molnar2022Model-agnosticLearning}.

Considering the variety of ways in which explanations can be derived, their evaluation depends on the intended purpose of use. For example, one can assess how effectively humans utilize explanations \cite{Ehsan2021OperationalizingAI,Wang2019DesigningAI}, or evaluate the explanatory function itself by measuring aspects such as size or sparsity \cite{Yang2017ScalableLists, Ustun2016SupersparseSystems, Claassen2013LearningNP-hard}. Additionally, it is possible to quantify certain aspects related to explanations, such as evaluating the extent to which explanations predict model outputs \cite{Chen2022Use-Case-GroundedEvaluation, Lakkaraju2016InterpretableSets}. For benchmarking purposes, a common practice is the use of synthetic data with a known data-generating process. This approach facilitates the comparison of actual explainability with expected explanations in various contexts.

\subsection{Who needs explanation?}

The demand for \gls{ML} explanations is pertinent across various sectors with the specific needs and objectives varying by domain and stakeholder diversity. Model explainability is not merely a desirable attribute but can be a crucial aspect for reasons ranging from model debugging to scientific exploration. This section delineates the roles of key stakeholders and the significance of explanations beyond performance within their respective domains.

Model creators, typically the developers of ML models, find interpretability crucial for debugging tasks \cite{Bhatt2020ExplainableDeployment}. It is important to know how the model relies on features to make predictions in order to fix unexpected behavior. For instance, the model creator might be interested in a model that makes decisions based on meaningful features rather than sensitive features in order to enhance generalizability or fairness. Such scrutiny cannot be achieved by only observing performance.

Operators, who use a model's outputs in their tasks, also require an understanding of the decision-making rationale. For instance, classifying a patient in a hospital into a particular health status should not be particularly helpful. It could be more useful to investigate the conditions that have contributed to this \cite{Razavian2015Population-levelFactors}, and this becomes even more crucial in the event of legal matters. Additionally, in the education domain, understanding why a student might drop out could be more valuable than just predicting it \cite{Pellagatti2021GeneralizedDropout,Berens2019EarlyMethods}. This is because, as in medicine, the treatment depends on the probable cause.

The people who are subject of decision-making also have to get a kind of explanation. For instance, a loan approval model may recommend the rejection of an applicant based on specific financial variables. Understanding the rationale behind a decision empowers the applicant to make informed future choices or contest an unjust or biased decision. Regulatory examiners, often working in regulated industries, expect similar explanations. They are responsible for auditing \gls{ML} models to ensure compliance with industry standards and ethical norms \cite{Chen2023Globally-ConsistentEvaluation, Flores2016FalseBlacks}. 

Finally, data analysts are increasingly utilizing \gls{ML} models for tasks aimed at understanding data-generating processes in both industrial and scientific research contexts \cite{Freiesleben2022ScientificPhenomena, SilvaFilho2023AAchievement}. These models often supplant traditional statistical methods due to their flexibility in handling large volumes of data without requiring prior domain knowledge. Although ML models can offer high predictive accuracy, they may lack explanatory power, thereby impeding a comprehensive understanding of the phenomena under investigation. Incorporating interpretability can address this limitation by elucidating the relationships among variables, consequently facilitating hypothesis generation for subsequent research.

\subsection{Explanations scope}

In addition to the types of model explanation techniques and the nature of the data, explainers can be categorized based on their scope. Local explainers refer to individual predictions, while global explainers quantify the average behavior of a model. Specifically, explanations can be further categorized into feature effects, which are commonly expressed through graphs, and feature importance, which provides score-based summary measures of the overall contribution of features.

Feature effects describe how the impact of a feature varies across its value range, using a simplified function derived from \(f\), $ g: Xs \xrightarrow{} Y$, being $Xs$ a set of features to be explained with a size typically of 1 or 2 features. Examples of global feature effects are \gls{ME}\cite{Long2021UsingOutcomes, Mize2019AModels}, \gls{ALE} plots \cite{Apley2020VisualizingModels}, \gls{PD} plots \cite{Friedman2001GreedyMachine.} and \gls{SHAP} \cite{10.5555/3295222.3295230}. 

Score-based explanations, often referred to as feature importance, essentially provide a ranking of features based on how much each one decreases the model's prediction error. The most commonly used methods are the tree-based \gls{MDI} \cite{Breiman2001RandomForests} and the model-agnostic \gls{PFI}\cite{Fisher2018AllSimultaneously} and its variations \cite{Molnar2023Model-agnosticApproach, Strobl2008ConditionalForests}. Additionally, there are score-based versions of some feature effects techniques such as \gls{PD} \cite{Greenwell2018AMeasure}, \gls{ME} \cite{long1997regression} and \gls{SHAP} \cite{Lee2023SHAPForecasting}.

\subsection{Can XAI really obtain knowledge about the world?}
\label{xai_can_fail_but_is_usable}

Discussing the cability of \gls{XAI} to extract knowledge from the world is essential, given that the core argument of this thesis hinges on \gls{XAI} being an invaluable tool for deriving trustworthy insights. This view is in line with the growing trend among researchers towards more transparent \gls{AI} and \gls{ML} models, as a response to their increasing integration in society \cite{Arya2019OneTechniques}. These researchers advocate that the empirical use of \gls{ML} could pivot scientific research towards a theory-independent method, allowing data to convey its own story without pre-existing hypotheses about the data, as noted in various studies \cite{Kitchin2014BigShifts, Anderson2008TheObsolete, Naimi2014BigThink, Andrews2023TheIdeal, Lieberson2008ImplicationSciences}.

While models known for their inherent transparency have faced minimal criticism, post-hoc techniques encounter more scrutiny  despite their widespread use across various fields such as education \cite{Lezhnina2022CombiningPISA, Martinez-Abad2020EducationalAssessment}, healthcare \cite{Jauhiainen2021NewAthletes, Stiglic2020InterpretabilityHealthcare}, social science \cite{Berger2023ExplainableChains, Bellantuono2023DetectingIntelligence}, and sociology \cite{Li2023ApplyingPandemic, Fan2023InterpretableInequality}.


The primary critique stems from the potential mismatch between what the opaque model is doing and what the post-hoc model attempts to explain \cite{Rudin2019StopInstead, Mullainathan2017MachineApproach, Babic2021BewareCare}. On the other hand, \cite{Sullivan2022UnderstandingModels} argues that gaining real-world knowledge with ML models is feasible as long as the link between model and phenomenon uncertainty can be assessed. Similarly, \cite{Cichy2019DeepModels} and \cite{Zednik2021SolvingIntelligence} suggest that \gls{XAI} can aid in understanding the real world, but they remain vague about how the model and phenomenon are connected.

Through the lens of philosophy of science and epistemology, authors in \cite{Fleisher2022UnderstandingAI} draw parallels between \gls{XAI} and the fundamental concepts of understanding. While there is some disagreement in the field, there is consensus that understanding is not an all-or-nothing state. Genuine understanding comes in degrees and can accommodate some inaccuracy and falsehood. In other words, understanding can still be valid even if the information or concepts it's based on are not entirely accurate. This ties into the concept of \textit{idealization} in scientific models, which refers to the process of simplifying or abstracting certain aspects of a phenomenon or model to make it more tractable \cite{Jebeile2015ExplainingIdealizations}. Building on this, \cite{Fleisher2022UnderstandingAI} argues that \gls{XAI} research has a solid foundation in science and promising avenues.

Rudin and colleagues \cite{Rudin2019StopInstead} advocate for the use of inherently interpretable models rather than combining opaque models with post-hoc techniques, especially in high-stakes decisions. Although this may initially seem like a criticism of post-hoc techniques, the critique centers around the inappropriate selection of models that are too complex without much performance improvement. Nevertheless, the use of inherent interpretable models does not prohibit the application of post-hoc methods. In fact, post-hoc methods are model-agnostic and can be applied to any model and can provide extra insights. As \cite{Molnar2022Model-agnosticLearning} notes, inherent interpretable models should always be included in benchmarks. 

In this context, it is posited that \gls{XAI}, particularly via post-hoc techniques, has the potential to augment model interpretability. Such enhancement is achieved either by providing additional insights into inherently interpretable models or by shedding light on functions of otherwise opaque models. And while these insights may not perfectly mirror the target model, they can be crucial in developing new theories based on real-world data. These theories can then be explored further to advance science and knowledge.
Therefore, results from ML explanations provide not an end goal, but the starting point for further analysis and conceptualization.


\subsection{Model-agnostic global explainers}
\subsubsection{ME Plots and Scores}

ME, or analytical derivative, was initially defined in the traditional statistical literature. The marginal term here is derived from the econometrics discipline as the "additional" effect, which has a  different mean from the rest of this thesis, where the marginal term is related to the probability distribution of an underlying feature.

The ME was established as an alternative to explain the coefficients of features in non-linear models, especially those entailing interactions that obscure the direct interpretation of coefficients. These more complex models lose their direct interpretation of coefficients, meaning that interpretation requires a first understanding of the details of the specified model\cite{Leeper2021InterpretingMargins, long1997regression}. The ME is also useful to inform the variable contribution in the natural scale on Generalized Linear Models (GLM), which involve transformations of the linear predictor into other discrete outcomes, such as logistic regressions, where coefficients typically lack direct interpretability and do not align with the scale of interest.

The ME effects of a variable $x_s$ are in the function of all other remaining variables $X_c$ and represent  for continuous variables the change in the probability when the $x_s$ varies in small change, as defined:

\begin{equation}
ME(X_s) = \lim_{{h \to 0}} \frac{{f(X_c|(x_s + h)) - f(X_c|x_s)}}{h}
\label{me}
\end{equation}

In practice, $h$ is the value of $x_s$ and the ME effects can be straightforwardly plotted over $x_s$ . However, usually, summary measures are the main unit of interest, such as:

\begin{itemize}
    \item Marginal Effect at the Mean (MEM) is simply the computation of the MEs around the mean of the feature distribution. In practice, MEM is close to the AME if \(f(X)\) is not too noisy.
\end{itemize}

\begin{itemize}
    \item ME at the Representative Value (MER) is a simplification of MEM calculation for a value that could be an interesting operation point for the research domain. The marginal effect is calculated for each variable at a particular combination of X values. Thus, MER provides a means to understand and communicate model estimates at theoretically important combinations of feature values.
\end{itemize}


\subsubsection{PD Plots and Scores}
\label{subpdp}

The Partial Dependence (PD) plots serve as a graphical representation that quantifies the effect of specific features on the predicted outcome within a supervised learning model while holding other variables constant (\textit{ceteris paribus}). These plots offer insights into the average marginal contribution of a feature of interest $x_s$ to the model's prediction, with the remaining features $X_c$ held constant. By doing so, if the predictive model closes the real world, \gls{PD}s allow a causal interpretation of the role of $x_s$ in the model if data meets the independence assumption \cite{Zhao2021CausalModels}. The underlying function can be mathematically described as follows:

\begin{equation}
PD(x_s)=\frac{1}{n}\sum_{i=1}^{n}f(x_s=j, X^{(i)}_{c})
\label{pd}
\end{equation}

where $f(x_s = j, X^{(i)}_{c})$  represents the model's predicted output when the feature  $x_s$ is intervened upon to assume a specific value $j$, while the remaining features $X_{i_{c}}$ are held their observed values in the dataset. The value $j$ is drawn from the marginal distribution of $x_s$. To be plotted, $j$ assumes values within a defined grid of the ordered $x_s$ where \ref{pd} is computed. For categorical features, $j$ assumes each category as a possible value.

A more specific method for estimating PD is utilizing Individual Conditional Expectation (ICE) curves. ICE curves \cite{Goldstein2015PeekingExpectation} provide a distinct curve $PD(x_s)$ for each individual data point $i$ in the sample. Essentially, the PD is computed as the average of these ICE curves. This granular decomposition facilitated by ICE allows for identifying potential interaction effects between $x_s$ and the remaining features $X_c$ at global level, which may not be observed when solely relying on \gls{PD}s. 

In an attempt to yield scores from \gls{PD}, \cite{Greenwell2018AMeasure} proposed a simple score considering that a feature's importance is inversely related to the flatness of its \gls{PD} Plot; a flatter \gls{PD} plot suggests lesser importance, while greater variation in the \gls{PD} indicates higher significance. Ass \gls{PD} ignores feature relationships, this \gls{PD}-based score captures only the main effect of the feature and ignores potential feature interactions


\subsubsection{ALE plots}

The Accumulated Local Effects (ALE) technique was established as an additional alternative to illustrate the feature effects. Distinct from prior methods, \gls{ALE} focuses on variations in predictions rather than the predictions themselves, thereby isolating individual feature effects. Also, \gls{ALE} is computed by parts of the data in an attempt to keep adherent to the data relationships without extrapolating. 

In the \gls{ALE} framework, intervals are theoretically defined as in \gls{ME}, using derivatives, but in practice, \gls{ALE} employs a grid \(Z\) based on quantiles of the feature of interest $X_s$ .  This process involves computing the effect of $X_s$ separately for each quantile $z$ intervening on $X_s$ twice: assuming the lower and upper quantile limits, while keeping all other variables, $x_c$, constant. The essence of \gls{ALE} lies in adjusting $x_s$ for all observations between these two bounds and calculating the change in the prediction function. This method seeks to capture the local effect (LE) of $x_s$ within the confines of each quantile, effectively isolating its influence by comparing the outcomes when data interventions are applied at the quantile's lower and upper limits.

Assuming data independence and a linear effect of $X_s$ within the quantile interval, the average differences in the predictions between the maximum and minimum interventions represent the isolated local effect of $X_s$, which is computed as:

\begin{equation}
\begin{aligned}
LE({X_s, z}) = \frac{1}{n}\sum_{i=1}^{n} f(x_s = x_{\text{max(z)}}, X_c) - f(x_s = x_{\text{min(z)}}, X_c)
\end{aligned}
\label{accAle}
\end{equation}


For visualization, this LE is subsequently accumulated over the grid $Z$. Theoretically, this accumulation is accomplished by integrating the expectations over intervals defined by the derivatives of the variable of interest. In practice, however, it can be estimated via summation across grid $Z$ as per Equation \ref{accAle}. Notably, this equation is also centered, ensuring that the average of $ALE(x_s)$ is zero with respect to the marginal distribution of $X_s$. 
\begin{equation}
\begin{aligned}
ALE(x_s) &= \sum_{{z \in Z, z}} LE(x_s, z) \\
&= ALE(x_s,z) - \frac{1}{n}\sum_{i=1}^{n} LE(x_s,z)
\end{aligned}
\label{centerAle}
\end{equation}

Estimating interaction effects requires a modification to the equations. For instance, in the case of two interactions (second-order), the intervals in \ref{accAle} have to change to rectangular regions. Additionally, in \ref{centerAle}, second-order \gls{ALE} require double-centering concerning both variables involved in the interaction (see details in \cite{Apley2020VisualizingModels}). Importantly, \gls{ALE} is not inherently suitable for analyzing categorical variables that lack ordinality.

\subsection{ALE Decomposition}

In linear models, the predictive function is a sum of the components that can be treated individually, the intercept, and the weight of each feature included in the function (\ref{lm}). The same can be applied to any high-dimensional function that can be decomposed into a sum of components of increasing dimensionality. In the following equation from \cite{Molnar2019QuantifyingInterpretability}, the predictive function is expressed as a sum of the intercept, individual (first order) feature effects, and interactions (second and higher order) effects.

\begin{eqnarray}\label{eqn:decomp} f(x)  = &\overbrace{f_0}^\text{Intercept} + \overbrace{\sum_{j=1}^p f_j(x_j)}^\text{1st order effects} + \overbrace{\sum_{j<k}^p f_{jk}(x_j, x_k)}^\text{2nd order effects} + \ldots + \overbrace{f_{1,\ldots,p}(x_1, \ldots, x_p)}^\text{p-th order effect}
%\\ =  & \sum_{S \subseteq \{1,\ldots,p\}} f_S(x_S)
\end{eqnarray}

Unlike other techniques, \gls{ALE} allows the function decomposition as unique components \cite{Apley2020VisualizingModels}. The \gls{ALE} components are computed conditional on the values of intervals and over the marginal distribution of all other features. This orthogonality-like property- called pseudo-orthogonality by the \gls{ALE} author, ensures that the main effects can indicate how each feature affects the prediction, independent of the values of the other feature. The interaction effect indicates the joint effect of the features, not considering the main effects of related features. 

\subsubsection{Global SHAP explanations}
\label{chap2_shap}

The aim of SHAP is to clarify individual model predictions by quantifying the contribution of each feature via the Shapley Values (SV) \cite{Shapley1953AGames}. Initially designed for local interpretability, SHAP can be seamlessly adapted for global model explanation by aggregating individual feature contributions \cite{Lundberg2020FromTrees.}. The most used SHAP-based score to indicate feature contribution is the absolute average of SV for each feature. Owing to its robust theoretical foundation and extensive implementations as software libraries, SHAP has emerged as a predominant technique in both industrial applications and academic research \cite{Bhatt2020ExplainableDeployment}.

The SV derives from coalitional game theory, where each feature value assumes the role of a player in a cooperative game, and the model prediction represents the total value or payout of the coalition. The SV is designed to allocate this collective payout equitably among the contributing features. Specifically, the SV $\phi_v(i)$ for a player $i$ with a characteristic function $v$ is computed as follows: 

\begin{equation}
\phi_v(i) = \sum_{S \subseteq N \backslash \{i\}} \frac{|S|! \left( |N| - |S| - 1 \right)!}{|N|!} \left( v\left( S \cup \{i\} \right) - v(S) \right)
\end{equation}

Where the summation iterates over all possible coalitions $S$ that exclude the player \(i\), thereby calculating the average additional contribution of player $i$ across all these coalitions. The term $|S|$ denotes the cardinality of coalition $S$ while $|N|$ indicates the cardinality of the complete set of players $N$. 

The fraction $\frac{|S|! \left( |N| - |S| - 1 \right)!}{|N|!}$ function as the weighting factor for each coalition $S$. It quantifies the number of ways to form $S$ and then adds $i$ relative to the total number of ways to form any coalition, including $i$. 

Finally,  the term$\left( v\left( S \cup \{i\} \right) - v(S) \right)$ calculates the additional contribution of player $i$ to coalition $S$. 

When accounting for all possible coalitions, SHAP assumes feature independence and integrates over the marginal distribution, akin to PD Plots and other permutation-based explainability techniques. Consequently, this approach introduces the issue of extrapolation.

SHAP  inherits axiomatic properties from SV, namely Efficiency, Symmetry, Dummy, and Additivity.

\begin{itemize}
    \item \textbf{Efficiency}, also termed as local accuracy in the SHAP context \cite{10.5555/3295222.3295230}, stipulates that the sum of SV for all features must equate to the total predictive value generated by the coalition of all features.
  
    \item \textbf{Dummy} axiom pertains to features that do not affect the model's prediction; such features are allocated a zero SV, reflecting their lack of contribution.

    \item \textbf{Additivity} or \textbf{Linearity} property is particularly relevant in post-hoc interpretability settings. It posits that the total attribution of a feature is the summation of all SVs associated with that feature across different models or scenarios.
\end{itemize}

In addition to these inherited properties, SHAP introduces unique attributes:

\begin{itemize}
    \item \textbf{Missingness} is designed to uphold the Efficiency property during the SHAP computation, especially when data may be incomplete or missing.
  
    \item \textbf{Consistency} ensures that the attribution of a feature changes in correlation with its SV. If a feature becomes more important, its attribution should increase correspondingly, and vice versa.
\end{itemize}

Consequently, SHAP can be represented as an additive feature attribution method.

\begin{equation}
g(z')=\phi_0+\sum_{v=1}^M\phi_v'
\end{equation}

Where $g$ is the explanation model, $z'\in\{0,1\}^M$ is the number of simplified input features - a binary vector indicates the presence or absence of a given feature within the coalition $S$.




\subsubsection{PFI scores}

The Permutation Feature Importance (\gls{PFI}) is a model-agnostic metric used to evaluate the contribution of each feature to the predictive power of a trained \gls{ML} model, \(f\). Given a feature matrix \(X\) and a target vector \(Y\), the \gls{PFI} for a particular feature is calculated by measuring the increase in a specified error measure \(L(Y, f)\) when the values of that feature are randomly permuted.

Let \(f: X \rightarrow Y\) be the trained model, where \(X \in \mathbb{R}^{n \times p}\) is the feature matrix with \(n\) samples and \(p\) features, and \(Y\) is the target space. The error measure \(L(Y, f)\) quantifies the discrepancy between the predicted and true target values. The \gls{PFI} of a given feature \(x_i\) is defined as follows:

\begin{equation}
\text{PFI}(x_i) = E\left[ L(Y, f(X)) - L\left(Y, f(x_{\text{-}i, \text{perm}})\right) \right]
\end{equation}

Here, \(x_{\text{-}i, \text{perm}}\) denotes the feature matrix \(X\) where the \(i\)-th feature column has been permuted randomly. The expectation \(E[\cdot]\) is taken over multiple permutations to obtain a stable estimate.

A higher \gls{PFI} value for a feature indicates a greater contribution to the model's predictive capability. Conversely, a low or negative \gls{PFI} suggests that the feature may be irrelevant or even detrimental to the model's performance. Usually, the \gls{PFI} values are normalized to be ranked. Typically, \gls{PFI} values are normalized and sorted such that they sum to one, to facilitate comparative ranking among the features. 


\subsubsection{MDI scores}

The Mean Decrease in Impurity (\gls{MDI}) is a metric specifically designed for assessing feature importance in tree-based models like Random Forests and Gradient Boosting Trees. As \gls{PFI}, \gls{MDI} is a loss-based metric and measures the average reduction in impurity—typically Gini impurity, entropy, or mean squared error—that a feature brings about when used for splitting in the decision trees that constitute the model.

For a given feature \( x_i \), its \( \text{MDI}(x_i) \) is defined as:

\begin{equation}
\text{MDI}(x_i) = \frac{1}{T} \sum_{t=1}^{T} \Delta I(t, x_i)
\end{equation}

where \( T \) is the total number of trees in the ensemble, and \( \Delta I(t, x_i) \) is the reduction in impurity in tree \( t \) attributable to feature \( X_i \).

The impurity reduction \( \Delta I(t, x_i) \) for a specific tree \( t \) and feature \( x_i \) is given by:

\begin{equation}
\Delta I(t, x_i) = \sum_{n \in \text{Nodes}(t, x_i)} w_n \Delta I_n
\end{equation}

where \( \text{Nodes}(t, x_i) \) is the set of nodes that use \( x_i \) for splitting in tree \(t\), \( w_n \) is the proportion of samples reaching node \(n\), and \( \Delta I_n \) is the impurity reduction achieved by the split at node \( n \).

As in \gls{PFI}, the \gls{MDI} values are often normalized and sorted to provide a ranking of feature importances.











