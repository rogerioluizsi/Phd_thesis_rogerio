\section{Summary}

This chapter illustrated the significance of the proposed metrics in a real-world application: Identify and track the relevance of features in secondary educational outcomes in the Brazilian public educational system. The application was defined as a repeated cross-sectional analysis, which required a definition of a new process since, to our knowledge,  there are no other examples of this trend analysis using supervised learning. 

The findings of this study may also provide valuable insights for researchers interested in Brazilian secondary education. While it is well-established in the literature that factors such as \textit{income}, \textit{age}, \textit{race}, and \textit{parent's education} have a strong impact on educational achievement, it is still an open question as to how these factors evolve and influence achievement over the period of study. Moreover, student computers at school, which presents mixed findings regarding its effects in the literature, have been highlighted in this chapter as one of the most effective policies regarding variables related to schools. Additionally, the knowledge extraction process leverages hypotheses that have either only been discussed qualitatively or not at all. For example, the study has suggested that improving \textit{Faculty education}, \textit{Faculty appropriate training} (especially for language teachers), and addressing Faculty workload could be important for improving secondary school achievement. Nevertheless, this investigation is not able to provide  causal conclusions, and further research by domain experts is needed to confirm the findings. 